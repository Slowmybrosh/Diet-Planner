\chapter*{}
%\thispagestyle{empty}
%\cleardoublepage

%\thispagestyle{empty}

\cleardoublepage
\thispagestyle{empty}

\begin{center}
{\large\bfseries Aplicación de dietas adaptativas: Adaptando tu alimentación a tus necesidades}\\
\end{center}
\begin{center}
Alejandro Olivares del Rey Pierres\\
\end{center}

%\vspace{0.7cm}
\noindent{\textbf{Palabras clave}: frontend, framework, git, backend, dataset, project management, crawler, design thinking}\\

\vspace{0.7cm}
\noindent{\textbf{Resumen}}\\

A todo el mundo nos ha ocurrido el problema de tener muchos ingredientes en la despensa, al final acabas olvidando algunos, como tomate escondido en el frigorífico o el típico tarro que tenemos al fondo de la despensa detrás de ingredientes que se terminan usando mucho más. Cuando se va a utilizar un ingrediente que sabemos que tenemos, y lo encontramos caducado e inutilizable sentimos una frustración por no recordar su existencia, además de tener que volver a comprarlo en el supermercado. Por ello, en este proyecto se analizará una posible solución a esta problemática por la que todos hemos pasado. Para un correcto desarrollo del proyecto, se utilizará una metodología ágil permitiendo obtener las ventajas del uso de este tipo de metodología. Se necesitarán perfilar los usuarios a los que va dirigida la solución del problema, haciendo uso del diseño orientado al usuario, pensando específicamente que quiere dicho usuario. Para guardar el progreso y llevar una correcta gestión se utilizarán herramientas de control de versiones. Además de analizar los diversos lenguajes que se pueden utilizar para el desarrollo del software, seleccionando el más adecuado dependiendo del tipo de implementación que se realice en el proyecto. 

Los contenidos incluidos en la memoria son:
\begin{enumerate}
    \item Introducción: Una breve introducción al los problemas alimentarios que existen en nuestro país y la descripción del problema a resolver.
    \item Estado del arte: Un vistazo a aplicaciones similares que puedan ser competencia directa con la solución que se va a desarrollar.
    \item Metodología: Descripción de la metodología a seguir en el proyecto, con las principales claves de la metodología ágil que se buscan seguir en el proyecto, las historias de usuario que han servido de guía a lo largo del mismo, user journeys, perfiles de usuarios objetivo de la aplicación y una descripción detallada de los objetivos a seguir. Además se incluye una descripción de design thinking que recoge el proceso seguido para el desarrollo del proyecto.
    \item Planificación. Se recoge la planificación del proyecto desde el inicio del proyecto hasta la elección de la metodología, se expone la dificultad de la planificación siguiendo una mentalidad ágil centrada en las necesidades del usuario, poniendo como punto de partida el desarrollo de la infraestructura del proyecto y como siguiente objetivo la funcionalidad para resolver el problema del usuario.
    \item Implementación: Sección dedicada a la justificación de cada decisión técnica tomada, desde el lenguaje del proyecto hasta el tipo de solución planteada, añadiendo también las decisiones relativas a la infraestructura del proyecto.
    \item Recetas. Explicación de la obtención de recetas, su la manera de guardarlas y acceder a ellas y explicación de la solución diseñada para encontrar recetas en la base de datos, añadiendo consideraciones iniciales de la funcionalidad que resuelve el problema a tratar, algoritmo utilizado y breve explicación de la implementación.
    \item Prototipo. Explicación del uso de prototipo, junto con el diseño de cada parte que lo compone desde la  interfaz, con un diseño inicial y la explicación de la implementación de la misma añadiendo pequeñas funcionalidades para aumentar la usabilidad de la misma, hasta el desarrollo del backend haciendo uso de un framework, estableciendo las conexiones entre el frontend y el backend.
    \item Pruebas. Descripción de los test realizados a las vistas de la aplicación que ven los usuarios para encontrar recetas y viajes por la aplicación de los usuarios descritos.
    \item Análisis del despliegue. Análisis de como desplegar la aplicación para que sea accesible para todos los usuarios desde el sofá de su casa y explicación del diseño de la infraestructura que se utilizará en el despliegue.
    \item Costes: Desglose de gastos en la producción de la aplicación
    \item Conclusiones: Consideraciones finales del proyecto y trabajos futuros.
\end{enumerate}
\cleardoublepage


\thispagestyle{empty}


\begin{center}
{\large\bfseries Adaptative diet app: Adapting your diet to your needs}\\
\end{center}
\begin{center}
Alejandro, Olivares del Rey Pierres \\
\end{center}

%\vspace{0.7cm}
\noindent{\textbf{Keywords}: frontend, framework, git, backend, dataset, project management, crawler, design thinking}\\

\vspace{0.7cm}
\noindent{\textbf{Abstract}}\\

Everyone has had the problem of having a lot of ingredients in the pantry, in the end you end up forgetting some, like tomatoes hidden in the fridge or the typical jar we have at the back of the pantry behind ingredients that end up being used much more. When we are going to use an ingredient that we know we have, and we find it expired and unusable we feel frustrated for not remembering its existence, in addition to having to buy it again at the supermarket. Therefore, this project will analyze a possible solution to this problem that we have all gone through. For a correct development of the project, an agile methodology will be used, allowing to obtain the advantages of the use of this type of methodology. It will be necessary to profile the users to whom the solution to the problem is directed, making use of the user-oriented design, thinking specifically about what the user wants. Version control tools will be used to keep the progress and to manage it correctly. In addition to analyzing the various languages that can be used for software development, selecting the most appropriate depending on the type of implementation to be carried out in the project.

The contents included in the memory are:
\begin{enumerate}
    \item Introduction: A brief introduction to the food problems that exist in our country and the description of the problem to be solved.
    \item State of the art: A look at similar applications that could be in direct competition with the solution to be developed.
    \Methodology: Description of the methodology to be followed in the project, with the main keys of the agile methodology to be followed in the project, the user stories that have served as a guide throughout the project, user journeys, profiles of target users of the application and a detailed description of the objectives to be followed. It also includes a description of design thinking that reflects the process followed for the development of the project.
    \item Planning. The planning of the project from the beginning of the project until the choice of the methodology, the difficulty of the planning is exposed following an agile mentality focused on the user's needs, putting as a starting point the development of the project infrastructure and as the next objective the functionality to solve the user's problem.
    \item Implementation: Section dedicated to the justification of each technical decision taken, from the language of the project to the type of solution proposed, adding also the decisions related to the project infrastructure.
    \item Recipes. Explanation of how to obtain recipes, how to save and access them and explanation of the solution designed to find recipes in the database, adding initial considerations of the functionality that solves the problem to be addressed, algorithm used and brief explanation of the implementation.
    \item Prototype. Explanation of the use of prototype, along with the design of each part that composes it from the interface, with an initial design and explanation of the implementation of the same adding small features to increase the usability of it, to the development of the backend using a framework, establishing the connections between the frontend and backend.
    \item Tests. Description of the tests performed to the views of the application that users see to find recipes and journeys through the application of the described users.
    \item Deployment analysis. Analysis of how to deploy the application so that it is accessible to all users from the sofa at home and explanation of the design of the infrastructure to be used in the deployment.
    \item Costs: Breakdown of expenses in the production of the application.
    \item Conclusions: Final considerations of the project and future work.
\end{enumerate}

\chapter*{}
\thispagestyle{empty}

\noindent\rule[-1ex]{\textwidth}{2pt}\\[4.5ex]

Yo, \textbf{Alejandro Olivares del Rey Pierres}, alumno de la titulación Grado en ingeniería informática de la \textbf{Escuela Técnica Superior
de Ingenierías Informática y de Telecomunicación de la Universidad de Granada}, con DNI 77166527D, autorizo la
ubicación de la siguiente copia de mi Trabajo Fin de Grado en la biblioteca del centro para que pueda ser
consultada por las personas que lo deseen.

\vspace{6cm}

\noindent Fdo: Alejandro Olivares del Rey Pierres

\vspace{2cm}

\begin{flushright}
Granada a 11 de Noviembre de 2023.
\end{flushright}


\chapter*{}
\thispagestyle{empty}

\noindent\rule[-1ex]{\textwidth}{2pt}\\[4.5ex]

D. \textbf{Juan Julián Merelo Guervós}, Profesor del Área de ATC del Departamento ICAR de la Universidad de Granada.

\vspace{0.5cm}

\textbf{Informan:}

\vspace{0.5cm}

Que el presente trabajo, titulado \textit{\textbf{Aplicación de dietas adaptativas,  Adaptando tu alimentación a tus necesidades}},
ha sido realizado bajo su supervisión por \textbf{Juan Julián Merelo Guervós}, y autorizamos la defensa de dicho trabajo ante el tribunal
que corresponda.

\vspace{0.5cm}

Y para que conste, expiden y firman el presente informe en Granada a 11 de Noviembre de 2023.

\vspace{1cm}

\textbf{Los directores:}

\vspace{5cm}

\noindent \textbf{Juan Julián Merelo Guervós}

\chapter*{Agradecimientos}
\thispagestyle{empty}

       \vspace{1cm}


A toda la gente que me ha apoyado en el desarrollo de este trabajo. Empezando por mis padres, que se leyeron la primera versión de esta memoria allá por Julio solo para ayudarme a corregir frases que no entendiesen o errores tipográficos. A mis amigos de Almería que siempre me han acompañado en los momentos difíciles. A los nuevos amigos que he hecho por el camino en el transcurso del desarrollo del proyecto. A Marien, que ha aguantado cada bloqueo mental que he tenido y sobretodo a Celia, la persona que más me ha apoyado en este viaje, aguantándome cuando algo no salía bien y había que rehacerlo, animándome cada vez que me encontraba derrotado. Y a mi tutor, JJ, por sus orientaciones en la realización de este trabajo.

