\chapter{Base de datos}
\begin{figure}[H]
    \begin{lstlisting}[style=python]
        RecipeIngredient.objects.all().delete()
        Recipes.objects.all().delete()
        Ingredients.objects.all().delete()
        
        configuration = Configuration()
        # Carga de los datos del CSV
        with open(configuration.data_ingredients, "r") as f:
            reader = csv.reader(f, delimiter=",")
            next(reader)
            for row in reader:
                ingredient = Ingredients(id=row[0],name=row[1])
                ingredient.save()
        
        # # Cargar las recetas en formato JSON
        with open(configuration.data_recipes, "r", encoding="utf-8") as f:
            recipes = json.load(f)
        
        for recipe in recipes:
            _steps = ";".join(recipe['Steps'])
            row = Recipes(name=recipe['Name'],steps=_steps,number_of_ingredients=None)
            row.save()
        
        for recipe in recipes:
            ingredients = recipe['Ingredients_ID']
            name = recipe['Name']
            resultado = Recipes.objects.get(name=name)
            id = resultado.id
        
            for ingredient in ingredients:
                result_ingredient = Ingredients.objects.get(id=ingredient)
                recipeingredient = RecipeIngredient(recipe_id=resultado,ingredient_id=result_ingredient)
                recipeingredient.save()
    \end{lstlisting}
    \caption{\Gls{script} para insertar los datos en la \gls{base}}
    \label{sni:insertar}
\end{figure}

\begin{figure}[H]
    \begin{lstlisting}[style=python]
        ## Actualizar tabla para incluir el numero de ingredientes de la receta
        for recipe in recipes:
            name = recipe['Name']
            resultado = Recipes.objects.get(name=name)
            id = resultado.id
            num_ingredients = RecipeIngredient.objects.filter(recipe_id=id).count()
            Recipes.objects.filter(id=id).update(number_of_ingredients=num_ingredients)
    \end{lstlisting}
    \caption{Continuación del \Gls{script} para insertar los datos en la \gls{base}}
    \label{sni:insertar-continuacion}
\end{figure}

\begin{figure}[H]
\begin{lstlisting}[style=python]
        def buscarIngredienteNombre(self,name: str):
        """
        Metodo que recupera una lista de ingredientes de la base de datos

        Parameters
        ----------
        name : str
            Nombre del ingrediente a buscar.

        Raises
        ------
        Error_DB : No hay conexion con la base de datos.

        Returns
        -------
        resultados : List<(id, name)>
            Lista de ingredientes encontrados
        """
        try:
            results = Ingredients.objects.filter(name__icontains=name).all()
            format_results = []
            for result in results:
                format_results.append((result.id,result.name))
            return format_results
        except:
            raise Error_DB("Error de base de datos: No hay conexion con la base de datos")
    \end{lstlisting}
    \caption{Método para buscar ingredientes por nombre en la \gls{base}}
    \label{sni:buscarIngredienteNombre}
\end{figure}

\begin{figure}[H]
    \begin{lstlisting}[style=python]
        def buscarIngredienteID(self,id: int):
        """
        Metodo que busca un ingrediente por identificador en la base de datos.

        Parameters
        ----------
        id : Int
            Identificador del ingrediente.

        Raises
        ------
        Error_DB : No hay conexion con la base de datos.

        Returns
        -------
        results : Tuple(id, name)
            Ingrediente encontrado
        """
        try:
            results = Ingredients.objects.get(id=id)
            return (results.id,results.name)
        except:
            raise Error_DB("Error de base de datos: No hay conexion con la base de datos")
    \end{lstlisting}
    \caption{Método para buscar ingredientes por ID en la \gls{base}}
    \label{sni:buscarIngredienteID}
\end{figure}
\begin{figure}[H]
    \begin{lstlisting}[style=python]
        def getIngredientes(self, id: int):
            """
            Metodo para obtener los ingredientes de una receta por ID

            Parameters
            ----------
            id : int
                Identificador de la receta

            Raises
            ------
            Error_DB : No hay conexion con la base de datos.

            Returns
            -------
            list_ingredients : list<str>
                Lista de nombres de ingredientes que lleva la receta
            """
            try:
                list_ingredients = Ingredients.objects.filter(recipeingredient__recipe_id=id).select_related('recipeingredient').values_list('name', flat=True)
                return list_ingredients
            except:
                raise Error_DB("Error de base de datos: No hay conexion con la base de datos")
    \end{lstlisting}
    \caption{Metodo para obtener los ingredientes de una receta por ID en la \gls{base}}
    \label{sni:buscarRecetaIngredientesID}
\end{figure}
\begin{figure}[H]
    \begin{lstlisting}[style=python]
        def buscarRecetasNombre(self, name : str):
        """
        Metodo que busca una lista de recetas por nombre en la base de datos.

        Parameters
        ----------
        name : str
            Nombre de la receta.

        Raises
        ------
        Error_DB : No hay conexion con la base de datos.

        Returns
        -------
        resultados : List<(Recipes)>
            Lista de recetas encontradas
        """
        try:
            return Recipes.objects.filter(name__icontains=name).all()
        except:
            raise Error_DB("Error de base de datos: No hay conexion con la base de datos")
    \end{lstlisting}
    \caption{Metodo para buscar una lista de recetas por nombre en la \gls{base}}
    \label{sni:buscarRecetaNombre}
\end{figure}
\begin{figure}[H]
    \begin{lstlisting}[style=python]
        def buscarRecetasID(self, id: int):
        """
        Metodo que busca una receta por identificador en la base de datos.

        Parameters
        ----------

        ID : Int
            Identificador de la receta.

        Raises
        ------

            Error_Buscador : La sentencia esta mal escrita.

        Returns
        -------
            
        resultados : Tuple(id,name,steps,number_of_ingredients)
            Lista de recetas encontradas
        """
        try:
            result = Recipes.objects.get(id=id)
            return (result.id,result.name,result.steps,result.number_of_ingredients)
        except:
            raise Error_DB("Error de base de datos: No hay conexion con la base de datos")
    \end{lstlisting}
    \caption{Metodo para buscar una lista de recetas por ID en la \gls{base}}
    \label{sni:buscarRecetaID}
\end{figure}
\begin{figure}[H]
    \begin{lstlisting}[style=python]
        def buscarRecetasIngredientesNumber(self, id: list[int]):
        """
        Metodo que busca una lista de recetas por numero de ingredientes y que contengan al menos un ingrediente de la lista en la base de datos.

        Parameters
        ----------
        id : List<Int>
            Identificadores de ingredientes a utilizar.

        Raises
        ------
        Error_DB : No hay conexion con la base de datos.

        Returns
        -------
            
        resultados : List<(Recipes)>
            Lista de recetas que cumplen la condicion
        """
        try:
            recipes = Recipes.objects.filter(
                Q(number_of_ingredients__lte=len(id)) & Q(recipeingredient__ingredient_id__in=id)
            ).distinct()
            return recipes.all()
        except:
            raise Error_DB("Error de base de datos: No hay conexion con la base de datos")
    \end{lstlisting}
    \caption{Metodo para buscar una lista de recetas por su numero de ingredientes y que contengan un ingrediente de la lista en la \gls{base}}
    \label{sni:buscarRecetaNumeroIngredientes}
\end{figure}
\begin{figure}[H]
    \begin{lstlisting}[style=python]
        def buscarRecetasIngredientes(self, id: int):
        """
        Metodo que recupera los ingredientes de una receta por identificador.

        Parameters
        ----------
        id : Int
            Identificador de receta

        Raises
        ------
        Error_DB : No hay conexion con la base de datos.

        Returns
        -------
        resultados : List<(ingredient_id,ingredient_name)>
            Lista de recetas que cumplen la condicion
        """
        try:
            receta_ingredientes = RecipeIngredient.objects.filter(recipe_id=id).values('ingredient_id', ingredient_name=F('ingredient_id__name'))
            resultado = []
            for receta in receta_ingredientes:
                resultado.append((receta['ingredient_id'],receta['ingredient_name']))
            return resultado
        except:
            raise Error_DB("Error de base de datos: No hay conexion con la base de datos")
    \end{lstlisting}
    \caption{Metodo para recuperar los ingredientes de una receta por su identificador de la \gls{base}}
    \label{sni:buscarRecetaIDIngredientes}
\end{figure}
\begin{figure}[H]
    \begin{lstlisting}[style=python]
        def getRecetasRandom(self, number: int):
        """
        Metodo para obtener recetas aleatorias de la base de datos

        Parameters
        ----------
        number : int
            Numero de recetas a obtener

        Raises
        ------
        Error_DB : No hay conexion con la base de datos.

        Returns
        -------
        recetas : list<(Recipes)>
        """
        try:
            items = list(Recipes.objects.all())
            random_items = random.sample(items,number)
            return random_items
        except:
            raise Error_DB("Error de base de datos: No hay conexion con la base de datos")
    \end{lstlisting}
    \caption{Metodo para recuperar recetas aleatorias de la \gls{base}}
    \label{sni:buscarRecetaAleatorias}
\end{figure}

\newpage
\begin{figure}[H]
\begin{lstlisting}[style=python]
    from decouple import config

    class Configuration:
        """Clase configuracion para modularizar el uso de la base de datos"""
        _instance = None
    
        def __new__(cls):
            """Metodo para asegurar el singleton"""
            if not cls._instance:
                cls._instance = super().__new__(cls)
            return cls._instance
        
        def __init__(self):
            """Constructor de clase"""
            self._db_name = config("DB_NAME")
            self._data_recipes = config("DATA_RECIPES")
            self._data_ingredients = config("DATA_INGREDIENTS")
        
        @property
        def db_name(self):
            """Getter nombre base de datos"""
            return self._db_name
    
        @property
        def data_recipes(self):
            """Getter ruta al fichero de recetas"""
            return self._data_recipes
        
        @property
        def data_ingredients(self):
            """Getter ruta al fichero de ingredientes"""
            return self._data_ingredients
    
\end{lstlisting}
\caption{Clase configuracion para las variables de entorno}
\label{sni:entorno}
\end{figure}