\chapter{Pruebas}
\begin{figure}[H]
    \begin{lstlisting}[style=consola]
        def setUp(self) -> None:
            self.buscador = Buscador()
            self.factory = RequestFactory()
            recipe1 = Recipes(name="test1", steps="test1")
            recipe2 = Recipes(name="test2", steps="test2")
            recipe3 = Recipes(name="test3", steps="test3")
            recipe4 = Recipes(name="test4", steps="test3")
            recipe5 = Recipes(name="test5", steps="test5")
            recipe1.save()
            recipe2.save()
            recipe3.save()
            recipe4.save()
            recipe5.save()
            ingredient1 = Ingredients(id=0,name="test1")
            ingredient1.save()
            contains = RecipeIngredient(recipe_id=recipe2,ingredient_id=ingredient1)
            contains.save()
    
        def test_call_view_index(self):
            response = self.client.get(reverse("index"))
            self.assertEqual(response.status_code, 200)
            self.assertTemplateUsed(response, 'buscador.html')
        
        def test_call_view_resultados_no_exists(self):
            response = self.client.get(reverse("resultados"), follow=True)
            self.assertRedirects(response, '/')
    
        def test_call_view_resultados(self):
            self.client.cookies['IDs'] = []
            response = self.client.post(reverse("resultados"),data={'metodo': 'ingredientes'})
            self.assertEqual(response.status_code,200)
            self.assertTemplateUsed(response, 'resultados.html')
    
        def test_call_view_guardadas(self):
            request = self.client.get(reverse("guardadas"))
            self.assertEqual(request.status_code, 200)
            self.assertTemplateUsed(request, 'resultados.html')
    
        def test_call_view_detalle(self):
            request = self.client.get(reverse("detalle_receta",args="1"))
            self.assertEqual(request.status_code, 200)
            self.assertTemplateUsed(request, 'receta.html')
    
        def test_data_in_detalle(self):
            request = self.client.get(reverse("detalle_receta",args="1"))
            self.assertTemplateUsed(request, 'receta.html')
            self.assertEqual(request.context['receta_id'], 1)
\end{lstlisting}
    \caption{Pruebas de la funcionalidad encontrar recetas}
    \label{sni:test}
\end{figure}

\begin{figure}[H]
    \begin{lstlisting}[style=python]
        from locust import HttpUser, task
        class MyLocustUser(HttpUser):

            def __init__(self, *args, **kwargs):
                super().__init__(*args, **kwargs)
                self.csrf_token = None

            def on_start(self):
                response = self.client.get("/")
                self.csrf_token = response.cookies["csrftoken"]
                print(self.csrf_token)

            def post_with_csrf_token(self, url, data):
                headers = {"X-CSRFToken": self.csrf_token}
                response = self.client.post(url, headers=headers, data=data)
                return response
            
            @task
            def resultado(self):
                self.on_start()
                cookie_value = json.dumps({"seleccionados": ["0", "1", "2", "3", "4", "5", "6", "7"]})
                self.client.cookies.set("IDs", cookie_value)
                response = self.post_with_csrf_token("/resultados/",{'metodo': "ingredientes"})

                # Valida la respuesta
                if response.status_code == 200:
                    print("validada")

            @task
            def detalle(self):
                self.on_start()
                self.client.get("/receta/44")
    \end{lstlisting}
    \caption{\Gls{script} para la prueba de esfuerzo de Locust}
    \label{sni:loadtest}
\end{figure}