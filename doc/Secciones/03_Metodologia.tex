\chapter{Descripción de la metodología}
En todo proyecto es necesaria una metodología que ayude a guiar el siguiente paso, ya que en realidad el paso más importante es el siguiente y no el primero. Por esta razón, en este capítulo se describirá la metodología utilizada así como la planificación del proyecto.

\section{Marco ágil}
Debido a las limitaciones de la gestión de proyectos, a finales de los 90 se empezó a utilizar una metodología ágil. Pero no fue hasta 2001 que un grupo de expertos desarrolla el ``Manifiesto por el Desarrollo Ágil de Software''. Donde se especificaron los doce principios clave para el desarrollo de software. A partir de ese punto, cada vez han ido surgiendo más modelos ágiles. Los 12 conceptos clave\cite{OBS2016} a seguir en este proyecto son\:
\begin{enumerate}
    \item Los cambios son bienvenidos
    \item Software funcional en un periodo corto de tiempo
    \item Facilidad para medir el progreso
    \item Desarrollo sostenible
    \item Excelencia técnica y buen diseño
    \item Simplicidad
    \item Adaptación a circunstancias cambiantes
\end{enumerate}

\section{Impacto de git en la metodología}
\emph{Git} se trata de una herramienta de control de versiones que surge de la necesidad de llevar un seguimiento del código fuente y sus actualizaciones. Es el núcleo de herramientas como \emph{Github} y \emph{Gitlab}. Cualquier equipo de desarrolladores, que cuente con más de un trabajador, debe hacer uso de una de estas herramientas.

Ambas herramientas proporcionan un marco para la gestión de proyectos. Permitiendo que diferentes equipos implementen concurrentemente \emph{features} en el software creando ramas a partir del código principal. Cualquier cambio realizado debe reflejarse con su correspondiente \emph{issue}, dotando de coherencia dichos cambios. Estos no solo sirven para reflejar cambios en el código o correcciones, en un proyecto muy grande también permiten llevar un seguimiento de en que trabaja cada miembro. Todos los \emph{issues} deben estar vinculados a un objetivo o \emph{milestone}. Estos últimos constituyen los entregables (producto mínimo viable) a alcanzar o, lo que es lo mismo, la planificación del proyecto.

Al comenzar todo proyecto, la orientación del mismo se reflejará como una serie de historias de usuario especificadas en \emph{issues} y la planificación haciendo uso de \emph{milestones}.

\section{User Journeys}

En primer lugar se tratará de resolver el problema de un estudiante que cuenta con un presupuesto ajustado al mes y no puede permitirse el lujo de comprar en el supermercado todos los días. Dicho estudiante se siente frustrado por no ser capaz de controlar sus gastos en el supermercado y que ciertos ingredientes que no utiliza asiduamente le caduquen en la despensa. Por casualidad descubre el software desarrollado en este proyecto y se lo descarga. Una vez descargado abre la aplicación e introduce los ingredientes que tiene en su despensa. El estudiante podrá encontrar a su disposición diferentes recetas que realizar con los ingredientes introducidos. Una vez seleccionada la receta, el usuario puede comprobar los detalles de la misma.

Otro problema que se tratará de resolver es el que tiene una persona mayor que no se encuentra en disposición de ir al supermercado cada vez que le faltan ingredientes y tiene una despensa limitada. De pronto escucha la existencia de la aplicación y se dispone a probarla. Después de descargarla, con dificultad debido a sus achaques de la edad y a su dificultad para entender las nuevas tecnologías, la abre. En primer lugar debe introducir los ingredientes que tiene por casa y aunque encuentra cierta dificultad sustituir una nota de papel, con esfuerzo consigue hacer un inventario de sus ingredientes. La aplicación en este momento le podrá empezar a sugerir recetas con los ingredientes que tiene en su despensa. Después de elegir una receta, puede comprobar los detalles de la misma con detenimiento.


\section{Planificación}
En todo proyecto es muy importante saber que se está haciendo en todo momento y lo que se hará a continuación. Es uno de los puntos, y ventajas, más importantes de la metodología ágil. Por ello, en esta sección se especificará la planificación del proyecto\:
\begin{enumerate}
    \item Establecer el alcance.
    \item Inicio del desarrollo.
    \item Encontrar recetas.
    \item Sustituir ingredientes.
\end{enumerate}

En el marco de la planificación se han especificado los \emph{milestones}, detallados en el repositorio del proyecto, constituyendo cada uno un entregable (PMV) del mismo.

\subsection{Establecer el alcance}
En todo proyecto, ya sea desarrollo software o construir un edificio necesita una planificación. Por ello es muy importante especificar el problema a resolver. En este caso, facilitar a los usuarios el proceso de encontrar recetas que se adapten a los ingredientes que tienen en su despensa. En primer lugar, se han especificado los objetivos a seguir durante el desarrollo del proyecto. Sentar las bases del proyecto ayuda a tener una orientación sobre que hacer en cada momento. 

\subsection{Inicio del desarrollo}
Antes de lanzarse a programar hay que hacer una serie de análisis previos y resolver ciertas cuestiones algunas de ellas son:
\begin{enumerate}
    \item ¿Para que plataforma está orientado el software?
    \item ¿En que lenguaje se realizará?
    \item ¿Se usará una herramienta de control de versiones?
\end{enumerate}
Este paso en la planificación es extremadamente importante, ya que determinará el marco de desarrollo bajo un lenguaje específico que no se puede escoger a la primera de cambio ya que cada uno cuenta con una serie de ventajas y desventajas a tener en cuenta en el proyecto. Aunque estas cuestiones serán resueltas en un capítulo posterior. 

\subsection{Encontrar recetas}
Este es el primer objetivo del software a desarrollar, que permitirá ayudar a los usuarios definidos anteriormente a encontrar recetas que contengan los ingredientes con los que cuentan el usuario. Además, en este punto es necesario definir que tipo de ingredientes son los que tiene cualquier persona en su casa, como por ejemplo: Agua, sal o aceite. Y conforme avance el desarrollo de este objetivo habrá que resolver ciertas cuestiones, como la anterior, que vayan surgiendo. 

\subsection{Sustituir ingredientes}
También se planifica una de las posibles mejoras a seguir una vez se haya completado el desarrollo del proyecto. En este caso se permitiría que se recomendasen recetas a las que les falta uno o varios ingredientes pero es posible sustituirlos de manera coherente con otros ingredientes con los que sí cuenta el usuario. Pero esto es solo una posible mejora para el futuro del software, teniendo que analizar la viabilidad una vez completado el objetivo anterior. 