\chapter{Descripción de la metodología}
En todo proyecto es necesaria una metodología que ayude a guiar el siguiente paso, ya que en realidad el paso más importante es el siguiente y no el primero. Por esta razón, en este capítulo se describirá la metodología utilizada así como la planificación del proyecto.

\section{Marco ágil}
Debido a las limitaciones de la gestión de proyectos, a finales de los 90 se empezó a utilizar una metodología ágil. Pero no fue hasta 2001 que un grupo de expertos desarrolla el ``Manifiesto por el Desarrollo Ágil de Software''. Donde se especificaron los doce principios clave para el desarrollo de software. A partir de ese punto, cada vez han ido surgiendo más modelos ágiles. Los 12 conceptos clave\cite{OBS2016} a seguir en este proyecto son\:
\begin{enumerate}
    \item Los cambios son bienvenidos
    \item Software funcional en un periodo corto de tiempo
    \item Facilidad para medir el progreso
    \item Desarrollo sostenible
    \item Excelencia técnica y buen diseño
    \item Simplicidad
    \item Adaptación a circunstancias cambiantes
\end{enumerate}

\section{Impacto de git en la metodología}
\emph{Git} se trata de una herramienta de control de versiones que surge de la necesidad de llevar un seguimiento del código fuente y sus actualizaciones. Es el núcleo de herramientas como \emph{Github} y \emph{Gitlab}. Cualquier equipo de desarrolladores, que cuente con más de un trabajador, debe hacer uso de una de estas herramientas.

Ambas herramientas proporcionan un marco para la gestión de proyectos. Permitiendo que diferentes equipos implementen concurrentemente \emph{features} en el software creando ramas a partir del código principal. Cualquier cambio realizado debe reflejarse con su correspondiente \emph{issue}, dotando de coherencia dichos cambios. Estos no solo sirven para reflejar cambios en el código o correcciones, en un proyecto muy grande también permiten llevar un seguimiento de en que trabaja cada miembro. Todos los \emph{issues} deben estar vinculados a un objetivo o \emph{milestone}. Estos últimos constituyen los entregables (producto mínimo viable) a alcanzar o, lo que es lo mismo, la planificación del proyecto.

Al comenzar todo proyecto, la orientación del mismo se reflejará como una serie de historias de usuario especificadas en \emph{issues} y la planificación haciendo uso de \emph{milestones}.

\section{User journeys}

En primer lugar se tratará de resolver el problema de un estudiante que cuenta con un presupuesto ajustado al mes y no puede permitirse el lujo de comprar en el supermercado todos los días. Dicho estudiante se siente frustrado por no ser capaz de controlar sus gastos en el supermercado y que ciertos ingredientes que no utiliza asiduamente le caduquen en la despensa. Por casualidad descubre el software desarrollado en este proyecto y se lo descarga. Una vez descargado abre la aplicación e introduce los ingredientes que tiene en su despensa. El estudiante podrá encontrar a su disposición diferentes recetas que realizar con los ingredientes introducidos. Una vez seleccionada la receta, el usuario puede comprobar los detalles de la misma.

Otro problema que se tratará de resolver es el que tiene una persona mayor que no se encuentra en disposición de ir al supermercado cada vez que le faltan ingredientes y tiene una despensa limitada. De pronto escucha la existencia de la aplicación y se dispone a probarla. Después de descargarla, con dificultad debido a sus achaques de la edad y a su dificultad para entender las nuevas tecnologías, la abre. En primer lugar debe introducir los ingredientes que tiene por casa y aunque encuentra cierta dificultad sustituir una nota de papel, con esfuerzo consigue hacer un inventario de sus ingredientes. La aplicación en este momento le podrá empezar a sugerir recetas con los ingredientes que tiene en su despensa. Después de elegir una receta, puede comprobar los detalles de la misma con detenimiento.

\section{Perfiles de usuarios}
\subsection{Estudiante}
\begin{itemize}
    \item Nombre: Daniel Pérez
    \item Edad: 22 años
    \item Ocupación: Estudiante de Bioquímica
    \item Idiomas: Español, inglés y francés
    \item Descripción general: Daniel cursa el tercer año de su carrera en Bioquímica en una universidad pública, ansioso por frenar el hambre en el mundo, investigando una manera de crear cultivos que arraiguen en climas extremadamente secos. Divide su tiempo entre las clases, trabajos y amigos. Lee multitud de investigaciones científicas actuales, estando muy interesando en este mundo.
    \item Dispositivos: Daniel utiliza asiduamente el ordenador y vive enganchado al teléfono.
    \item Entorno social: Las amistades de Daniel son también estudiantes que están acabando sus estudios y se plantean iniciarse en el mundo laboral. Se dedican a diferentes campos no solo científicos. Pero todos están muy concienciados con las causas sociales. En el ámbito familiar, Daniel tiene una relación cercana con sus padres y su hermano Jorge. Realizan actividades habituales en familia y están todo el tiempo que pueden juntos. Aunque él viva en un piso para estudiantes con tres personas más.
    \item Motivación: Daniel busca reducir el derroche de alimentos que se ponen malos en su piso. Ya que en ocasiones se olvidan de algún ingrediente que se termina poniendo malo. Además, cuenta con unos recursos realmente limitados. 
    \item Cita: Erradicar el hambre en el mundo es una responsabilidad compartida.
\end{itemize}

\subsection{Persona mayor}
\begin{itemize}
    \item Nombre: María Encarnación Sánchez
    \item Edad: 85 años
    \item Ocupación: Jubilada
    \item Idiomas: Español
    \item Descripción general: Maria Encarnación vive en un pequeño barrio al norte de Granada, donde las calles son demasiado empinadas para ir tirando del carrito de la compra por los inmensos escalones hasta el supermercado y vuelta. Muchos días piensa que ya no tiene edad para estar yendo a comprar cada dos o tres días, aunque no tenga mucho más que hacer. Se pasa las mañanas realizando tareas del hogar. Aunque cuando encuentra un hueco, se arregla para ir a su parroquia, ubicada a tan solo unas pocas calles de su casa. Además, con la edad cada vez es más olvidadiza y cada poco se tiene que poner a buscar el papel de la receta para comprobar los ingredientes o pasos que no recuerda.
    \item Dispositivos: Aunque María Encarnación tiene un ordenador que le regaló su hijo, no sabe usarlo más que para buscar cosas en internet. Y el móvil que tiene es tan pequeño que apenas ve lo que pone en la pantalla. Pero el dispositivo que más utiliza se trata de una tablet muy antigua que le regalaron por navidad y usa todos los días mientras se mece en la butaca.
    \item Entorno social: El círculo más cercano de María Encarnación son sus amigas de la iglesia, su hijo voló del nido hace más de cincuenta años y su marido falleció hace tan solo 2 años. Cuando ocurrió se refugió en la Fe cristiana para sobrellevar la pérdida. Por ello acude cuando puede a rezar por el alma de su marido y tiene un círculo cercano de señoras de su edad. 
    \item Motivación: María Encarnación ya no tiene edad para ir a comprar cada vez que le falta algún ingrediente y su memoria no es lo que era, no recuerda muchas recetas de su juventud.
    \item Cita: La edad puede nublar la memoria de nuestras recetas favoritas, pero nunca desvanece el sabor de los momentos compartidos
\end{itemize}

\section{Planificación}
En todo proyecto es muy importante saber que se está haciendo en todo momento y lo que se hará a continuación. Es uno de los puntos, y ventajas, más importantes de la metodología ágil. Por ello, en esta sección se especificará la planificación del proyecto\:
\begin{enumerate}
    \item Inicio del desarrollo
    \item Encontrar recetas
    \item Despiegue de la solución
    \item Sustituir ingredientes
\end{enumerate}