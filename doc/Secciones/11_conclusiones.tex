\chapter{Conclusiones}

\section{Futuro del proyecto}
El proyecto no acaba con esta memoria, sino que se han marcado objetivos futuros para aumentar la utilidad de la aplicación, se han planteado dos objetivos futuros.
\begin{itemize}
    \item Desplegar la infraestructura en la nube de manera práctica
    \item Sustituir ingredientes dinámicamente para personas con alérgenos en las recetas
\end{itemize}

Si bien es cierto que desplegar la infraestructura supone un coste elevado, supone una ventaja frente a contar con los servidores de la aplicación \gls{on-premises}. Pero habría que analizar la manera en la que se puede obtener beneficio de la aplicación para costear su mantenimiento en la nube.

El segundo objetivo a alcanzar pretende solucionar la problemática que tienen algunos usuarios de alterar las recetas originales debido a alguna intolerancia o alergia alimentaria. La aplicación permitirá que al seleccionar los ingredientes deseados por el usuario, las recetas que contengan ingredientes problemáticos se sustituirán por ingredientes coherentes en la receta, pero también se buscarán recetas a las que se le puedan sustituir ingredientes originales por sustitutos seleccionados por el usuario.

\section{Conclusiones}
En conclusión, este trabajo de fin de grado ha abordado la gestión de un proyecto siguiendo una mentalidad ágil, en el que se ha llevado a cabo el diseño y desarrollo de una solución completamente orientada al usuario desde la definición del problema hasta la entrega de una solución final. Para ello se ha tenido que realizar una investigación en la que se ha necesitado investigar las metodologías actuales para este tipo de trabajos. 

Se superó el problema de la ausencia de unos datos de recetas en un formato correcto, analizando diferentes soluciones para extraer los datos. Desarrollando un robot que permitiera extraer las recetas de una fuente fiable y almacenándolas en un fichero estandarizado para ser cargado en una base de datos.

Además, se ha diseñado una solución a la problemática que tenían los usuarios perfilados en el apartado de metodología. Una vez completada la solución, no se quiso dejar como una biblioteca de funciones que recuperen información de una base de datos sino que al estar el proyecto tan orientado al usuario final, era necesario llevar la solución hasta ellos.

Para que el usuario final pudiera acceder y utilizar la aplicación se ha creado una interfaz como prototipo, que se irá mejorando en futuras iteraciones, después de escuchar el feedback que tengan que aportar. La interfaz actual es simple, con colores vivos que acentúen los rasgos minimalistas de la página web. Se utiliza una paleta de colores anaranjados, para dar calidez a la página. 

También era necesario conectar la \gls{interfaz} diseñada con la funcionalidad desarrollada inicialmente. Por ello, se eligió un \gls{framework} que facilitó la implementación de los puntos de acceso a la aplicación y la posibilidad de gestionar la aplicación web como conjunto. 

Por último, se ha analizado el despliegue en una infraestructura en la nube de \gls{AWS} y proponiendo una solución de infraestructura que permita mantener una alta disponibilidad, sin escatimar en seguridad. Aunque el coste inicial que se tiene que aportar para el despliegue en esta infraestructura es demasiado alto como para poderse probar. 

Entre los desafíos que he encontrado durante el viaje que supone el desarrollo del proyecto, sin duda el más complicado ha sido seguir una metodología orientada al usuario, ya que, durante el grado no se ha abarcado este tema en profundidad o con un enfoque práctico, a excepción de una asignatura. En el desarrollo de código, me he valido de los conocimientos adquiridos a lo largo de estos cinco años en el grado, aprovechando el contenido de las asignaturas orientadas al desarrollo web y al diseño de software.

Para terminar, a nivel personal, este proyecto ha sido una experiencia enriquecedora que ha fortalecido tanto mi conocimiento como mis habilidades en los ámbitos de gestión de proyectos y desarrollo de aplicaciones web. 