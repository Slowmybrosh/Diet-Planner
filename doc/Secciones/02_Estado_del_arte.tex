\chapter{Estado del arte}
Después de una investigación inicial en este campo buscando en diversos blogs de nutrición como \emph{HealthLine}\cite{Healthline2022} y revistas como \emph{Bussines Insider}\cite{BusinessInsider2021} y \emph{The Guardian}, donde se habla de aplicaciones que permiten desde desarrollar una dieta utilizando recetas publicadas por la comunidad o simplemente con los ingredientes que tenemos en nuestro frigorífico. 

Una de las más interesantes de las que se trata es \href{https://www.yummly.com}{\emph{Yummly}}, una aplicación disponible tanto en \emph{iOS} como en \emph{Android} que se adapta a las necesidades individuales de cada persona, incluyendo intolerancias alimentarias \cite{TheGuardian2016}. Esta ofrece un plan gratuito que permite buscar recetas, recomendaciones personalizadas, una lista de la compra. Además cuenta con un plan de pago que cuesta 4,99\$/mes, que incluye la creación de dietas personalizadas, información nutricional de las recetas, un motor de búsqueda por ingredientes que permite filtrar por ingredientes, alergias o tipo de dieta. Pero, desafortunadamente, no está disponible en España. 

Una aplicación quizá más conocida es \href{https://cookpad.com/es/home}{\emph{Cookpad}}, esta es una de las aplicaciones más descargadas de \emph{Apple Store} y \emph{Google Play} en la sección de cocina. Una de sus características principales es su comunidad, donde es posible compartir recetas y fotos de tus creaciones. Es posible usar esta aplicación de manera gratuita pero cuenta con algunas limitaciones. Por otra parte, la versión pro cuesta 2,99\$/mes y entre las funcionalidades que ofrece se encuentran planes de menús semanales elaborados por nutricionistas.

Otra de las aplicaciones más interesantes que se encontraron fue \href{https://www.eatthismuch.com/}{EatThisMuch}.Esta es una aplicación web, también disponible para \emp{Android} e \emph{iOS}, que permite generar una dieta de manera automática basándose en:
\begin{itemize}
    \item Las calorías que se quieren ingerir al día.
    \item El número de comidas que se desea realizar.
    \item El tipo de dieta que se quiere generar.
\end{itemize}
\emph{Eat This Much} tiene una versión gratuita abierta al usuario. Al generar la dieta es posible cambiar el plato aleatoriamente entre los que se recomiendan. Pero, el principal problema se basa en que la personalización de la dieta es muy general, sin tener en cuenta las intolerancias del usuario. De la misma manera, ofrecen una suscripción orientada a profesionales y entrenadores permitiendo que estos recomienden generen las dietas a sus clientes de manera totalmente personalizada. Esta suscripción cuesta 79,00\$/mes y solo incluye servicio para diez clientes, cada cliente extra costará 4\$/mes. 

La última aplicación de la que se hablará es \href{https://happyforks.com/}{\emph{HappyForks}}, que se trata de una herramienta que permite monitorizar una gran cantidad de valores nutricionales. Contando con una gran cantidad de utilidades para controlar una dieta ajustándose a las necesidades del usuario y una gran base de datos llena de recetas. Una de las utilidades que ofrecen permite analizar recetas y productos mostrando los ingredientes para no causar alergias al usuario. 

El proyecto que se busca desarrollar tiene como objetivo crear un software que permita a los usuarios encontrar recetas de manera sencilla utilizando los ingredientes que ya tienen en su despensa. Esto facilita tener todas las recetas reunidas en un solo lugar, organizadas y disponibles en todo momento. Además, el software también puede ayudar a los usuarios a aprender nuevos estilos de cocina y experimentar con nuevos ingredientes. Sobre el modelo de negocio, se debería analizar detenidamente si contará con una suscripción mensual con diversas ofertas o si se liberará añadiendo anuncios a la aplicación con la posibilidad de eliminarlos mediante una pequeña transacción. 