\chapter{Estado del arte}
Después de una investigación inicial en este campo buscando en diversos blogs de nutrición como \emph{HealthLine}\cite{Healthline2022} y revistas como \emph{Bussines Insider}\cite{BusinessInsider2021} y \emph{The Guardian}, donde se habla de aplicaciones que permiten desde desarrollar una dieta utilizando recetas publicadas por la comunidad o simplemente con los ingredientes que tenemos en nuestro frigorífico. 

Una de las más interesantes de las que se trata es \href{https://www.yummly.com}{\emph{Yummly}}, una aplicación disponible tanto en \emph{iOS} como en \emph{Android} que se adapta a las necesidades individuales de cada persona, incluyendo intolerancias alimentarias \cite{TheGuardian2016}. Esta ofrece un plan gratuito que permite buscar recetas, recomendaciones personalizadas, una lista de la compra. Además cuenta con un plan de pago que cuesta 4,99\$/mes, que incluye la creación de dietas personalizadas, información nutricional de las recetas, un motor de búsqueda por ingredientes que permite filtrar por ingredientes, alergias o tipo de dieta. Pero, desafortunadamente, no está disponible en España. 

Una aplicación quizá más conocida es \href{https://cookpad.com/es/home}{\emph{Cookpad}}, esta es una de las aplicaciones más descargadas de \emph{Apple Store} y \emph{Google Play} en la sección de cocina. Una de sus características principales es su comunidad, donde es posible compartir recetas y fotos de tus creaciones. Es posible usar esta aplicación de manera gratuita pero cuenta con algunas limitaciones. Por otra parte, la versión pro cuesta 2,99\$/mes y entre las funcionalidades que ofrece se encuentran planes de menús semanales elaborados por nutricionistas.

Otra de las aplicaciones más interesantes que se encontraron fue \href{https://www.eatthismuch.com/}{EatThisMuch}.Esta es una aplicación web, también disponible para \emp{Android} e \emph{iOS}, que permite generar una dieta de manera automática basándose en:
\begin{itemize}
    \item Las calorías que se quieren ingerir al día.
    \item El número de comidas que se desea realizar.
    \item El tipo de dieta que se quiere generar.
\end{itemize}
\emph{Eat This Much} tiene una versión gratuita abierta al usuario. Al generar la dieta es posible cambiar el plato aleatoriamente entre los que se recomiendan. Pero, el principal problema se basa en que la personalización de la dieta es muy general, sin tener en cuenta las intolerancias del usuario. De la misma manera, ofrecen una suscripción orientada a profesionales y entrenadores permitiendo que estos recomienden generen las dietas a sus clientes de manera totalmente personalizada. Esta suscripción cuesta 79,00\$/mes y solo incluye servicio para diez clientes, cada cliente extra costará 4\$/mes. 

La última aplicación de la que se hablará es \href{https://happyforks.com/}{\emph{HappyForks}}, que se trata de una herramienta que permite monitorizar una gran cantidad de valores nutricionales. Contando con una gran cantidad de utilidades para controlar una dieta ajustándose a las necesidades del usuario y una gran base de datos llena de recetas. Una de las utilidades que ofrecen permite analizar recetas y productos mostrando los ingredientes para no causar alergias al usuario. 

El GED indica la cantidad de calorías que un individuo debe consumir para abastecer los requerimientos energéticos. Si el gasto corresponde a la ingesta de alimentos, mantendrá su peso. De lo contrario, si la ingesta supera al requerimiento, ganará peso. El gasto energético no solo corresponde a actividades fisiológicas, sino que también depende de las actividades físicas que desarrolle el individuo. Se deben tener en cuenta los siguientes mecanismos: 
\begin{itemize}
    \item Metabolismo basal
    \item Actividad dinámica específica de los alimentos
    \item Termogénesis inducida por el ejercicio
\end{itemize}

El metabolismo basal se trata de la cantidad de energía necesaria para realizar los procesos vitales estando en reposo. Una manera de calcularlo es multiplicando 24kcal por el peso del individuo en kilogramos. El resultado sería la cantidad de energía que necesitaría una persona para sobrevivir 24 horas estando completamente en reposo. 

La actividad dinámica específica de los alimentos, se refiere a la cantidad de calorías que quema una persona al masticar digerir y absorber los alimentos. El valor oscila entre un 6\% y un 10\% dependiendo de cuanto se haya ingerido y del gasto energético basal.

Por último, hay que tener en cuenta el consumo de calorías cuando se hace ejercicio, ya que en los esfuerzos físicos se consume gran cantidad de energía. Dependiendo del nivel de actividad del individuo, se consumirán más calorías. 
\begin{table}[h]
    \begin{center}
        \begin{tabular}{| c | r |}
            Actividad & Gasto energético \\ \hline
            Sedentario & 1,2 \\
            Leve & 1,375 \\
            Moderada & 1,55 \\
            Intensa & 1,725 \\
            Extrema & 1,9 \\ \hline
        \end{tabular}
        \caption{Tabla 1: factor de actividad física}
        \label{tab:FAF}
    \end{center}
\end{table}

La fórmula de Harris-Benedict aprovecha los parámetros introducidos anteriormente, además de otros valores individuales como la edad (años), peso (kg), talla (cm), etc... Para calcular el gasto energético en reposo. Se debe tener en cuenta el sexo del individuo \cite{GER}
\begin{equation}
    Varones:GER = 66,5+(13,75*peso)+(5,08*talla)-(6,78*edad)
\end{equation}
\begin{equation}
    Mujeres:GER = 65,51+(9,56*peso)+(1,85*talla)-(4,68*edad)
\end{equation}

Otra fórmula que permite calcular el GER es la ecuación de Mifflin-St Jeor, desarrollada en 1990, es considerada más precisa que la ecuación Harris-Benedict. También considera el peso (kg), altura (cm), edad (años) y el sexo del individuo \cite{CARRASCO2007}
\begin{equation}
    Varones:GER = (10 * peso) + (6.25 * altura) - (5 * edad) + 5
\end{equation}
\begin{equation}
    Mujeres:GER = (10 * peso) + (6.25 * altura) - (5 * edad) - 161
\end{equation}

Por otra parte, la fórmula de Katch-McArdle solo consta de una ecuacuón tanto para hombres, mujeres, edad y estatura. Esto simplifica el cálculo del metábolismo básal usando solo la cantidad de masa magra. En primer lugar se debe conocer el valor de grasa que tiene el individuo. Si, por ejemplo, tiene un 22\% de grasa entonces consta de un 78\% de masa magra (kg) que se sustituirá en la ecuación \cite{Katch-McArdle}
\begin{equation}
    TMB = 370 + (21,6 * masa)
\end{equation}.

Hay que tener en cuenta que cada fórmula está calculando el gasto energético en reposo, a esto hay que multiplicarle el factor de actividad que se expone en la \autoref{tab:FAF}.