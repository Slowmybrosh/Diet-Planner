\chapter{Planificación}
En este tercer capítulo se narrará la planificación seguida en el proyecto, desde la definición del problema a resolver hasta la especificación de las metodologías que se utilizarán en el proyecto. 

\section{Introducción}
En el apartado de introducción se detalla la motivación del proyecto, exponiendo los motivos que llevan al desarrollo del proyecto, además se determina que se considera como ``dieta equilibrada'', ya que no todo el mundo tiene el mismo concepto y, adicionalmente, que se considera una ``dieta variada''. 

Por otra parte, se describe el problema desde el punto de vista de los clientes de la aplicación. El problema que se va a resolver es la necesidad que tiene el usuario para encontrar nuevas recetas, o recordar antiguas, utilizando los ingredientes a su disposición. 

Una vez descrito el problema, fue necesario describir a quien se dirige la solución, se determinó que existen dos tipos de cliente a los que se resuelve la necesidad:
\begin{enumerate}
    \item Estudiante
    \item Persona mayor
\end{enumerate}

\section{Estado del arte}
En este capítulo se definen las principales soluciones existentes similares a lo que se busca desarrollar. Se realizó una búsqueda de aplicaciones que pudieran ser parecidas o con una funcionalidad que sea una solución al problema planteado. 

Se encontraron algunas aplicaciones, en blogs de nutrición y revistas \cite{Healthline2022}\cite{TheGuardian2016}\cite{BusinessInsider2021}, que cumplen los requisitos especificados antes, la mayoría son de pago o imponen restricciones a la hora de escoger ingredientes que se utilicen en la búsqueda de recetas. 

Se han detallado las más relevantes o parecidas que se han encontrado.

\section{Metodologías}
En esta sección inicialmente se describe el concepto de marco ágil y las ventajas que se aprovecharán en el desarrollo del proyecto. 

Para el proyecto se plantea el uso de \emph{\gls{design}}, realizando una pequeña investigación leyendo diversos artículos académicos\cite{wolniak2017design}\cite{thoring2011understanding} que han servido de guía para obtener un entendimiento del concepto de \emph{\gls{design}} y sus etapas en el desarrollo. Además, se ha especificado el avance en las primeras etapas del proyecto.

La etapa de empatizar con el usuario objetivo se completa pensando en las necesidades que hay que cubrir del usuario. Especificando el viaje del usuario (\emph{\gls{journey}}) en la aplicación. Un \emph{\gls{journey}} es una herramienta que permite representar paso a paso el proceso que necesita el usuario para alcanzar su objetivo, en este caso encontrar una receta que se adapte a él. \cite{gasparini2015perspective}

Para definir a los usuarios se utiliza la metodología de persona, que permite definir al usuario por medio de una serie de información planteada por medio de preguntas para conocer a fondo al cliente, tanto sus necesidades como sus motivaciones. 

Las preguntas propuestas para el perfilado del usuario permiten entender las necesidades de los clientes potenciales y abasteciéndolos de las funcionalidades o servicios que necesitan.

El proyecto más complejo que el desarrollar una aplicación, es muy importante llevar una correcta gestión del mismo. Para llevar a cabo esa gestión se utiliza \emph{\Gls{git}}, una herramienta de control de versiones que permite trabajar siguiendo una mentalidad ágil resolviendo poco a poco \emph{\glspl{issue}}.

Existen diferentes herramientas que permiten el uso de \emph{\Gls{git}}, se utiliza \emph{\Gls{github}} por ser la herramienta con la que he trabajado y tengo mayor experiencia en su uso. Pero existen otras herramientas como \emph{\Gls{jira}} o \emph{\Gls{gitlab}} en las que tengo menos experiencia.

En la gestión del proyecto se han afrontado algunos desafíos como la correcta creación de \emph{\glspl{issue}} que definan de manera corta un problema o desafío que se necesita resolver para continuar el desarrollo. La definición de \emph{\glspl{milestone}} y que se espera de cada hito también ha supuesto un desafío en esta etapa del proyecto, debido a la poca experiencia que poseía al abordar el proyecto con una mentalidad ágil.

En un punto del desarrollo del \emph{\gls{milestone}} se desarrolló una parte del proyecto de manera prematura, utilizando una rama para ello. Para afrontar el problema de manera organizada, se abrieron \emph{\gls{pr}} lo más atómicos posibles con el objetivo de cerrar los \emph{\glspl{issue}} relativos a este milestone, dejando pendientes los prematuros. Una vez completados todos los \emph{\glspl{issue}} se tomó la decisión de dejar la rama abierta sin mezclar y extraer los cambios realizados para cerrar los problemas abordados. Para ello, se utiliza la operación \emph{cherry-pick}, siendo mejor opción que eliminar la rama y tener que repetir los cambios para solventar los \emph{\glspl{issue}}. Desde una nueva rama con los cambios extraídos de la problemática, se mezclaron con la rama principal sin inconvenientes.

En la planificación del proyecto se tomó la decisión de especificar únicamente los primeros \emph{\glspl{milestone}} del proyecto debido a la imposibilidad de conocer el camino que tomará el desarrollo siguiendo una mentalidad ágil.