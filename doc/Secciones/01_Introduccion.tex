\chapter{Introducción}
En este capítulo se describirá la motivación y los objetivos principales del proyecto.

\section{Motivación}
En la actualidad, debido a una gran concienciación por parte de diferentes organismos orientados la salud, ha aumentado la preocupación por mantener una alimentación saludable y equilibrada. Sin embargo, la gran mayoría de personas no tienen los conocimientos suficientes como para elaborar una dieta adaptada a sus necesidades individuales. Ya no solo estamos hablando de personas que buscan bajar peso, sino de personas que tienen algún tipo de intolerancia o alergia alimentaria. 

Según la Fundación de Seguridad Alimentaria (FSA), en el mundo existen unas 520 millones de personas con este tipo de problemas alimentarios. Solo en España se estima que entre el 1-3\% de los adultos y un 4-6\% de los niños padecen algún inconveniente al consumir determinados alimentos. Además, debemos diferenciar a las personas con intolerancias alimentarias de las que sufren alergias. Ya que no tienen los mismos efectos aunque a primera vista sean parecidos. Las alergias se producen cuando el sistema inmune entra en contacto con el alérgeno, mientras que las intolerancias están ligadas a la dificultad para digerir determinados alimentos.\cite{FSA}

Lo que se plantea es crear una aplicación que ayude al usuario en la tarea de planear que comer, creando una dieta completa y totalmente equilibrada adaptada completamente a sus necesidades. Que no solo se ajuste simplemente a un objetivo de calorías especificado, sino que te recomiende las recetas para llevarla a cabo. Además, ajustando la receta para sustituir los ingredientes que puedan provocar algún inconveniente al usuario.

\section{Objetivos}
\begin{enumerate}
    \item Diseñar e implementar un algoritmo que identifique y sustituya ingredientes en una receta según las intolerancias alimentarias del usuario y mantenga la coherencia de la receta.
    \begin{enumerate}
        \item Desarrollar una base de datos de ingredientes y alternativas para los casos de intolerancias alimentarias.
        \item Implementar una lógica que permita la identificación y sustitución de ingredientes incompatibles con las intolerancias del usuario.
    \end{enumerate}
    \item Desarrollar un sistema de gestión de dietas personalizadas que tenga en cuenta las necesidades de cada usuario, incluyendo sus parámetros antropométricos y restricciones alimentarias.
    \begin{enumerate}
        \item Implementar la capacidad de crear y almacenar perfiles de usuario con información relevante, como edad, peso, altura e intolerancias alimentarias.
        \item Diseñar e implementar un algoritmo que genere dietas equilibradas basandose en las necesidades y características individuales del usuario.
    \end{enumerate}
    \item Implementar un sistema de gestión de usuarios que permita la creación, autenticación y autorización de usuarios en el software. 
    \begin{enumerate}
        \item Diseñar e implementar una base de datos de usuarios con sus respectivas credeniales e información personal. 
        \item Desarrollar un mecanismo de autenticación y autorización para garantizar la seguriadad y privacidad de los datos de los usuarios.
    \end{enumerate}
\end{enumerate}