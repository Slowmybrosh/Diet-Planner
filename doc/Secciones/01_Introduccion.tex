\chapter{Introducción}
En este capítulo se describirá la motivación y los objetivos principales del proyecto.

\section{Motivación}
En la actualidad, debido a una gran concienciación por parte de diferentes organismos orientados la salud, ha aumentado la preocupación por mantener una alimentación saludable y equilibrada. Sin embargo, la gran mayoría de personas no tienen los conocimientos suficientes como para elaborar una dieta adaptada a sus necesidades individuales. Ya no solo estamos hablando de personas que buscan bajar peso, sino de personas que tienen algún tipo de intolerancia o alergia alimentaria. 

 En el mundo existen unas 520 millones de personas con este tipo de problemas alimentarios. Solo en España se estima que entre el 1-3\% de los adultos y un 4-6\% de los niños padecen algún inconveniente al consumir determinados alimentos \cite{alimentacion}. Además, debemos diferenciar a las personas con intolerancias alimentarias de las que sufren alergias. Ya que no tienen los mismos efectos aunque a primera vista sean parecidos. Las alergias se producen cuando el sistema inmune entra en contacto con el alérgeno, mientras que las intolerancias están ligadas a la dificultad para digerir determinados alimentos.

Lo que se plantea es crear una aplicación que permita eliminar la carga de planear que comer, permitiendo al usuario crear una dieta completa y totalmente equilibrada adaptada completamente a sus necesidades. Que no solo se ajuste simplemente a un objetivo de calorías especificado, sino que te recomiende las recetas para llevarla a cabo. Además, ajustando la receta para sustituir los ingredientes que puedan provocar algún inconveniente al usuario.

\section{Objetivos}
\begin{itemize}
    \item Diseñar e implementar un software que permita a los usuarios la creación de una dieta completamente personalizada basada en una serie de parámetros. Teniendo en cuenta posibles intolerancias o alergias que tenga el usuario. 
    \item Diseñar y desplegar una solución en la nube para alojar el software desarrollado. Brindando escalabilidad y alta disponibilidad a los usuarios. 
\end{itemize}