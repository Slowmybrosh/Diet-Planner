\chapter{Introducción}

\section{Motivación}
En la actualidad, debido a una gran concienciación impartida por diferentes organismos orientados a la salud, ha aumentado la preocupación por mantener una alimentación saludable y equilibrada. Sin embargo, la gran mayoría de personas no tienen los conocimientos suficientes como para elaborar una dieta adaptada a sus necesidades individuales. Ya no solo estamos hablando de personas que buscan bajar peso, sino de personas que tienen algún tipo de intolerancia o alergia alimentaria.

Según la Fundación de Seguridad Alimentaria (FSA), en el mundo existen unas 520 millones de personas con este tipo de problemas alimentarios. Solo en España se estima que entre el 1-3\% de los adultos y un 4-6\% de los niños padecen algún inconveniente al consumir determinados alimentos. Además, debemos diferenciar a las personas con intolerancias alimentarias de las que sufren alergias. Ya que no tienen los mismos efectos aunque a primera vista sean parecidos. Las alergias se producen cuando el sistema inmune entra en contacto con el alérgeno, mientras que las intolerancias están ligadas a la dificultad para digerir determinados alimentos.\cite{FSA}

Cualquier persona sabe qué es una dieta, ya que todos nos alimentamos cada día. La palabra ``dieta'' se refiere al conjunto de alimentos que ingerimos. Sin embargo, no todo el mundo lleva una dieta equilibrada. Existen múltiples tipos de dietas, como omnívora, vegetariana, vegana, paleo, mediterránea, entre otras, las cuales se diferencian por los alimentos que se consumen.
Es un error común pensar que simplemente al comer de manera saludable se está siguiendo una dieta equilibrada. Para lograr una dieta equilibrada, es necesario consumir todos los nutrientes esenciales que el cuerpo necesita para mantener una salud óptima. Una dieta equilibrada se caracteriza por ser variada, moderada y proporcionada.

Cuando hablamos de variedad, nos referimos a la inclusión de alimentos de todos los grupos: frutas, verduras, cereales, legumbres, carnes, pescados, etc. Por supuesto, hay dietas que excluyen ciertos alimentos; en esos casos, se debe buscar una fuente alternativa de nutrientes. Por otro lado, una dieta debe ser moderada, lo que significa que se deben ingerir las cantidades necesarias de nutrientes sin excederse.

\section{Descripción del problema}
A todo el mundo nos ocurre el tener un montón de ingredientes en la despensa sin saber que hacer con ellos, que se terminan acumulando o caducando, haciendo que sea un derroche de dinero. El problema que se quiere resolver con el desarrollo de esta aplicación es el mal aprovechamiento de los ingredientes que se tienen por casa y el desconocimiento de determinadas recetas que solventen esa mala gestión.

El proyecto va dirigido a usuarios que cuentan con recursos económicos limitados, como un estudiante con un presupuesto ajustado, o personas con movilidad reducida que no puedan permitirse acudir al supermercado cada vez que les falte un ingrediente. 

Pero el uso de esta aplicación no está limitado a estos usuarios descritos, sino que puede ser utilizada por cualquier persona que quiera aprovechar los ingredientes que tengan sueltos por la cocina.

En primer lugar, necesitamos conocer a fondo el problema descrito, para comprender toda la complejidad que pueda tener. Pensando un poco es posible encontrar mucha profundidad a la hora de resolver el problema, por ejemplo, que se entiende por ingrediente o que ingredientes tiene todo el mundo en su casa. Estas preguntar irán surgiendo a medida que avance el desarrollo del proyecto.

Para resolver el problema descrito se utilizarán las mejores prácticas dentro de un enfoque ágil.

\section{Objetivos}
\begin{enumerate} 
    \item Búsqueda básica de recetas. El usuario debe ser capaz de encontrar recetas utilizando los ingredientes que desee.
\end{enumerate}