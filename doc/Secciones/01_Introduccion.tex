\chapter{Introducción}
En este capítulo se describirá la motivación y los objetivos principales del proyecto.

\section{Motivación}
En la actualidad, debido a una gran concienciación impartida por diferentes organismos orientados a la salud, ha aumentado la preocupación por mantener una alimentación saludable y equilibrada. Sin embargo, la gran mayoría de personas no tienen los conocimientos suficientes como para elaborar una dieta adaptada a sus necesidades individuales. Ya no solo estamos hablando de personas que buscan bajar peso, sino de personas que tienen algún tipo de intolerancia o alergia alimentaria.

Según la Fundación de Seguridad Alimentaria (FSA), en el mundo existen unas 520 millones de personas con este tipo de problemas alimentarios. Solo en España se estima que entre el 1-3\% de los adultos y un 4-6\% de los niños padecen algún inconveniente al consumir determinados alimentos. Además, debemos diferenciar a las personas con intolerancias alimentarias de las que sufren alergias. Ya que no tienen los mismos efectos aunque a primera vista sean parecidos. Las alergias se producen cuando el sistema inmune entra en contacto con el alérgeno, mientras que las intolerancias están ligadas a la dificultad para digerir determinados alimentos.\cite{FSA}

Cualquier persona sabe qué es una dieta, ya que todos nos alimentamos cada día. La palabra "dieta" se refiere al conjunto de alimentos que ingerimos. Sin embargo, no todo el mundo lleva una dieta equilibrada. Existen múltiples tipos de dietas, como omnívora, vegetariana, vegana, paleo, mediterránea, entre otras, las cuales se diferencian por los alimentos que se consumen.
Es un error común pensar que simplemente al comer de manera saludable se está siguiendo una dieta equilibrada. Para lograr una dieta equilibrada, es necesario consumir todos los nutrientes esenciales que el cuerpo necesita para mantener una salud óptima. Una dieta equilibrada se caracteriza por ser variada, moderada y proporcionada.

Cuando hablamos de variedad, nos referimos a la inclusión de alimentos de todos los grupos: frutas, verduras, cereales, legumbres, carnes, pescados, etc. Por supuesto, hay dietas que excluyen ciertos alimentos; en esos casos, se debe buscar una fuente alternativa de nutrientes. Por otro lado, una dieta debe ser moderada, lo que significa que se deben ingerir las cantidades necesarias de nutrientes sin excederse.

Además, la proporción adecuada de nutrientes es clave en una dieta equilibrada. Esto implica mantener un equilibrio entre macronutrientes (proteínas, grasas y carbohidratos) y micronutrientes (vitaminas y minerales) adecuado para las necesidades individuales de cada persona.

El objetivo es desarrollar una aplicación orientada a aquellas personas que tienen dificultad para llevar una dieta equilibrada. Esta herramienta permitirá a los usuarios crear dietas semanales personalizadas compuestas de recetas detalladas basadas en sus gustos, obtenidos durante el proceso de registro. Además, se contemplará la posibilidad de sustituir ingredientes problemáticos para aquellos que tengan alguna alergia o intolerancia, por otros igualmente coherentes que sí puedan consumir.

\section{Objetivos}
\begin{enumerate}
    \item Desarrollar un sistema de gestión de dietas personalizadas que tenga en cuenta las necesidades de cada usuario, incluyendo sus parámetros antropométricos y restricciones alimentarias.
    \begin{enumerate}
        \item Implementar la capacidad de crear y almacenar perfiles de usuario con información relevante, como edad, peso, altura e intolerancias alimentarias.
        \item Diseñar e implementar un algoritmo que genere dietas equilibradas basandose en las necesidades y características individuales del usuario.
        \item Implementar una función que permita a los usuarios reemplazar ingredientes problemáticos por alternativas adecuadas según sus alergias o intolerancias. Manteniendo una coherencia en la receta.
    \end{enumerate}
    \item Implementar un sistema de gestión de usuarios que permita la creación, autenticación y autorización de usuarios en el software. 
    \begin{enumerate}
        \item Diseñar e implementar una base de datos de usuarios con sus respectivas credeniales e información personal. 
        \item Desarrollar un mecanismo de autenticación y autorización para garantizar la seguriadad y privacidad de los datos de los usuarios.
    \end{enumerate}
\end{enumerate}