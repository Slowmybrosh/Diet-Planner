\chapter{Introducción}
En este capítulo se describirá la motivación y los objetivos principales del proyecto.

\section{Motivación}
En la actualidad, debido a una gran concienciación impartida por diferentes organismos orientados a la salud, ha aumentado la preocupación por mantener una alimentación saludable y equilibrada. Sin embargo, la gran mayoría de personas no tienen los conocimientos suficientes como para elaborar una dieta adaptada a sus necesidades individuales. Ya no solo estamos hablando de personas que buscan bajar peso, sino de personas que tienen algún tipo de intolerancia o alergia alimentaria.

Según la Fundación de Seguridad Alimentaria (FSA), en el mundo existen unas 520 millones de personas con este tipo de problemas alimentarios. Solo en España se estima que entre el 1-3\% de los adultos y un 4-6\% de los niños padecen algún inconveniente al consumir determinados alimentos. Además, debemos diferenciar a las personas con intolerancias alimentarias de las que sufren alergias. Ya que no tienen los mismos efectos aunque a primera vista sean parecidos. Las alergias se producen cuando el sistema inmune entra en contacto con el alérgeno, mientras que las intolerancias están ligadas a la dificultad para digerir determinados alimentos.\cite{FSA}

Cualquier persona sabe qué es una dieta, ya que todos nos alimentamos cada día. La palabra ``dieta'' se refiere al conjunto de alimentos que ingerimos. Sin embargo, no todo el mundo lleva una dieta equilibrada. Existen múltiples tipos de dietas, como omnívora, vegetariana, vegana, paleo, mediterránea, entre otras, las cuales se diferencian por los alimentos que se consumen.
Es un error común pensar que simplemente al comer de manera saludable se está siguiendo una dieta equilibrada. Para lograr una dieta equilibrada, es necesario consumir todos los nutrientes esenciales que el cuerpo necesita para mantener una salud óptima. Una dieta equilibrada se caracteriza por ser variada, moderada y proporcionada.

Cuando hablamos de variedad, nos referimos a la inclusión de alimentos de todos los grupos: frutas, verduras, cereales, legumbres, carnes, pescados, etc. Por supuesto, hay dietas que excluyen ciertos alimentos; en esos casos, se debe buscar una fuente alternativa de nutrientes. Por otro lado, una dieta debe ser moderada, lo que significa que se deben ingerir las cantidades necesarias de nutrientes sin excederse.

Además, la proporción adecuada de nutrientes es clave en una dieta equilibrada. Esto implica mantener un equilibrio entre macronutrientes (proteínas, grasas y carbohidratos) y micronutrientes (vitaminas y minerales) adecuado para las necesidades individuales de cada persona.

Es complicado decidir que hacer de comer, especialmente cuando se tienen ingredientes limitados o restricciones dietéticas específicas. Además suele ocurrir que las personas se olvidan de los detalles de una receta o perder la información relevante. por ello objetivo del proyecto es crear un recetario móvil que permita ayudar a las personas a llevar una dieta equilibrada encontrando recetas con el objetivo de optimizar el uso de los ingredientes en su despensa. En el caso de usuarios que tengan una intolerancia, alergia alimentaria o un tipo de dieta que no permita comer determinados alimentos, permitirá la sustitución coherente de los ingredientes problemáticos en la receta. Además, permitir que los usuarios guarden sus recetas favoritas para acceder rapidamente a ellas, pudiendo añadir notas a la receta para futuras referencias.

\section{Objetivos}
\begin{enumerate} 
    \item Busqueda básica de recetas. El usuario debe ser capaz de encontrar recetas utilizando los ingredientes que desee.
    \begin{enumerate}
        \item Sustitución de ingredientes. Implementar una característica que permita sugerir sustituciones adecuadas para los ingredientes faltantes en una receta, teniendo en cuenta las restricciones del usuario y las características de los ingredientes.
    \end{enumerate}
\end{enumerate}